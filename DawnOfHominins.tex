\chapter{Dawn of Hominins}
\subsection*{The Early Steps in Human Evolution}
Let's start by tracing our lineage back to very early beginnings. This chapter dives into the world of hominins. Before \textit{Homo sapiens} dominated the planet, several hominin species walked the Earth. The story of hominins begins millions of years back. In the paragraphs below, we will explore our most ancient ancestors.

The term \textit{hominin} refers to the evolutionary group that includes modern humans, our immediate ancestors, and other extinct species more closely related to us than to chimps. To truly understand our journey, it's crucial to start from the Miocene epoch, approximately 20 million years ago, when the ancestors of humans and chimpanzees, our closest living relatives, diverged from a common ancestor.

The discovery of \textit{Sahelanthropus tchadensis} in Chad, dating back to about 6-7 million years ago, introduces us to one of the oldest known hominins. Though the precise position of \textit{Sahelanthropus} in the human family tree remains debated, its discovery highlights the diverse features that early hominins possessed.

\subsubsection*{Appearance and Physical Features}
\textit{Sahelanthropus tchadensis} is known primarily from a single skull, which was discovered in Chad in 2001. Despite the limited material, several observations about its physical features can be made.

\paragraph{Cranial Capacity:} The brain size of \textit{Sahelanthropus} was small, akin to that of modern chimpanzees, with an estimated cranial capacity of around 320-380 cubic centimetres.

\paragraph{Face and Jaw:} One of the most striking features of the \textit{Sahelanthropus} skull is its flat face (orthognathic), which is more similar to later hominins than to apes. The prominent brow ridge (supraorbital torus) is another characteristic feature. The teeth, especially the canines, are relatively small and more human-like than ape-like.

\paragraph{Foramen Magnum Position:} Though \textit{Sahelanthropus}'s skull retains several primitive features, the position of the foramen magnum (the hole where the spinal cord exits the skull) suggests it might have been bipedal. This position is towards the skull's base, typically seen in bipedal creatures, implying an upright posture.

\subsubsection*{Behavior}
Given the scant fossil evidence, making definitive claims about the behaviour of \textit{Sahelanthropus tchadensis} is challenging. However, certain deductions can be made.

\paragraph{Bipedalism:} As mentioned earlier, the position of the foramen magnum suggests that \textit{Sahelanthropus} might have been bipedal. If this is true, it would have walked upright, at least part of the time, which would differentiate it from other apes and make it more similar to later hominins.

\paragraph{Diet:} The wear patterns and size of the teeth might suggest that \textit{Sahelanthropus} had a varied diet, which could include both plant material and possibly some meat.

\subsubsection*{Environment}
\textit{Sahelanthropus tchadensis} lived during a time when central Africa, including the region of Chad, was transitioning from a closed forested environment to a more open grassland setting. However, the specific area where the skull was found, known as the Djurab Desert today, was likely woodlands and lakes around 7 million years ago. Such environments would have offered a mix of resources, allowing for a diverse diet. The presence of other animal fossils found alongside \textit{Sahelanthropus}, like fish and antelopes, supports the idea of a varied environment with lakes or water bodies nearby.

Following \textit{Sahelanthropus}, species like \textit{Ardipithecus ramidus} emerged around 4.4 million years ago. "Ardi," as the most famous specimen is called, presents a mix of bipedal characteristics similar to humans and features more common in our primate ancestors. This indicates the early steps our lineage took towards bipedalism, a hallmark of human evolution.

The genus \textit{Australopithecus}, spanning from about 4 to 2 million years ago, marks a significant point in our evolutionary journey. Notably, the renowned "Lucy" (\textit{Australopithecus afarensis}) hailing from Ethiopia offers substantial insights. With her upright posture yet ape-like brain size, Lucy serves as a testament to the importance of bipedalism as an early evolutionary adaptation. Another species, \textit{Australopithecus sediba}, unearthed in South Africa, has showcased a blend of Australopithecine and early Homo traits, suggesting a possible transitional species.

The emergence of the \textit{Homo} genus around 2.5 million years ago signifies a notable shift. \textit{Homo habilis}, aptly named the "handyman," is believed to be among the first tool users. This adaptation, coupled with an increase in brain size, sets the stage for the rapid evolution that followed. Species like \textit{Homo erectus}, which emerged roughly 2 million years ago, are particularly significant. With their larger brain, \textit{erectus} not only developed more sophisticated tools but also became the first hominin to leave Africa, spreading across parts of Asia and Europe.

The evolutionary journey of hominins is not a straight path but rather a branching tree with multiple species co-existing and possibly even interacting. Throughout this odyssey, certain traits like bipedalism, tool use, and increased cognitive abilities defined the human lineage. These adaptations, driven by both environmental changes and complex biological processes, paved the way for the emergence of \textit{Homo sapiens}, i.e., us.

The Dawn of Hominins is a captivating story of resilience, adaptation, and evolution. By exploring our ancient ancestors, we not only uncover the roots of our species but also gain insights into the shared heritage that unites all of humanity. Every fossil uncovered and every bone studied adds a piece to the puzzle of our evolutionary history, reminding us of the remarkable journey that led to the world we know today.

