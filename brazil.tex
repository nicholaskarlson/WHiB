\section{Population Growth in Brazil: 1700 - 2020}

\subsection*{1700 - 1950 (Every 50 years)}
\begin{itemize}
    \item \textbf{1700}: Estimated to be around 300,000 (note that this is an approximation and mostly refers to the indigenous and colonial population).
    \item \textbf{1750}: Approximately 1.5 million.
    \item \textbf{1800}: Around 3 million.
    \item \textbf{1850}: Close to 7 million.
    \item \textbf{1900}: Approximately 17 million.
    \item \textbf{1950}: Roughly 52 million.
\end{itemize}

\subsection*{Commentary 1700 - 1950}
\begin{itemize}
    \item \textbf{1700-1800}: Slow population growth, with the area largely inhabited by indigenous peoples and a small number of Portuguese colonists.
    \item \textbf{1800-1900}: The population began to grow more significantly, influenced by the coffee boom, immigration, and the end of the slave trade.
    \item \textbf{1900-1950}: Rapid population growth, influenced by continued immigration, industrialization, and urbanization.
\end{itemize}

\subsection*{1950 - 2020 (Every 10 years)}
\begin{itemize}
    \item \textbf{1950}: Roughly 52 million.
    \item \textbf{1960}: Around 70 million.
    \item \textbf{1970}: Approximately 94 million.
    \item \textbf{1980}: Close to 121 million.
    \item \textbf{1990}: Approximately 146 million.
    \item \textbf{2000}: About 169 million.
    \item \textbf{2010}: Roughly 191 million.
    \item \textbf{2020}: Estimated to be around 212 million.
\end{itemize}

\subsection*{Commentary 1950 - 2020}
\begin{itemize}
    \item \textbf{1950-1980}: Rapid population growth continued, driven by high birth rates and declining mortality rates.
    \item \textbf{1980-2000}: Continued population growth, though the rate began to slow as birth rates started to decline.
    \item \textbf{2000-2020}: The population continued to grow but at a slower pace, reflecting further declines in birth rates and changes in demographic trends.
\end{itemize}

\subsection*{Overall Commentary}
\begin{itemize}
    \item From 1700 to 2020, the population of Brazil experienced significant growth, increasing from around 300,000 to about 212 million.
    \item The 19th and 20th centuries saw rapid population growth and significant social, economic, and political changes.
    \item The 21st century is marked by a transition to slower population growth, urbanization, and ongoing challenges and progress in various sectors.
\end{itemize}
