\section*{Population Growth in Canada: 1700 - 2020}

\subsection*{1700 - 1950 (Every 50 years)}
\begin{itemize}
    \item \textbf{1700}: Estimated to be around 20,000 (primarily Indigenous peoples and European settlers).
    \item \textbf{1750}: Approximately 90,000 (growth due to settlement and natural increase).
    \item \textbf{1800}: Around 250,000.
    \item \textbf{1850}: Close to 2.4 million.
    \item \textbf{1900}: Approximately 5.3 million.
    \item \textbf{1950}: Roughly 14 million.
\end{itemize}

\subsection*{Commentary 1700 - 1950}
\begin{itemize}
    \item \textbf{1700-1800}: Slow growth, primarily in settled areas of Eastern Canada.
    \item \textbf{1800-1900}: Significant growth due to immigration, westward expansion, and natural increase.
    \item \textbf{1900-1950}: Continued growth, influenced by immigration, economic development, and the baby boom post World War II.
\end{itemize}

\subsection*{1950 - 2020 (Every 10 years)}
\begin{itemize}
    \item \textbf{1950}: Approximately 14 million.
    \item \textbf{1960}: Around 17.9 million.
    \item \textbf{1970}: Approximately 21.4 million.
    \item \textbf{1980}: Close to 24.3 million.
    \item \textbf{1990}: Approximately 27.8 million.
    \item \textbf{2000}: About 30.7 million.
    \item \textbf{2010}: Roughly 33.5 million.
    \item \textbf{2020}: Estimated to be around 37.6 million.
\end{itemize}

\subsection*{Commentary 1950 - 2020}
\begin{itemize}
    \item \textbf{1950-1980}: Steady growth, influenced by immigration, economic prosperity, and the baby boom generation.
    \item \textbf{1980-2000}: Continued growth but at a slower rate, with increased diversity in immigration sources.
    \item \textbf{2000-2020}: Further growth and diversification, with an aging population and focus on sustainable development.
\end{itemize}

\subsection*{Overall Commentary}
\begin{itemize}
    \item Canada’s population has experienced significant growth and diversification over the past few centuries.
    \item The country faces challenges associated with an aging population, healthcare, and maintaining a balanced and sustainable growth.
    \item Efforts are ongoing to ensure equitable distribution of resources and opportunities across the vast geographic expanse of the country.
\end{itemize}