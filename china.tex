\section*{Population Growth in China: 1700 - 2020}

\subsection*{1700 - 1950 (Every 50 years)}
\begin{itemize}
    \item \textbf{1700}: Estimated to be around 100 million.
    \item \textbf{1750}: Approximately 225 million.
    \item \textbf{1800}: Around 300 million.
    \item \textbf{1850}: Close to 430 million.
    \item \textbf{1900}: Approximately 415 million (decline due to wars and rebellions).
    \item \textbf{1950}: Roughly 550 million.
\end{itemize}

\subsection*{Commentary 1700 - 1950}
\begin{itemize}
    \item \textbf{1700-1800}: Rapid growth, supported by agricultural expansion and stability under the Qing Dynasty.
    \item \textbf{1800-1900}: Growth and decline patterns due to internal strife, rebellions, and the impact of opium wars.
    \item \textbf{1900-1950}: Recovery and growth resumption, despite political instability, influenced by the end of imperial rule and the establishment of the People’s Republic in 1949.
\end{itemize}

\subsection*{1950 - 2020 (Every 10 years)}
\begin{itemize}
    \item \textbf{1950}: Approximately 550 million.
    \item \textbf{1960}: Around 667 million.
    \item \textbf{1970}: Approximately 818 million.
    \item \textbf{1980}: Close to 987 million.
    \item \textbf{1990}: Approximately 1.13 billion.
    \item \textbf{2000}: About 1.26 billion.
    \item \textbf{2010}: Roughly 1.34 billion.
    \item \textbf{2020}: Estimated to be around 1.41 billion.
\end{itemize}

\subsection*{Commentary 1950 - 2020}
\begin{itemize}
    \item \textbf{1950-1980}: Remarkable population growth, influenced by policies encouraging large families and later mitigated by the introduction of the One Child Policy in 1979.
    \item \textbf{1980-2000}: Slower growth rate due to the implementation of strict population control policies.
    \item \textbf{2000-2020}: Continued slow growth and aging population, with policy adjustments to address demographic challenges, including the introduction of the Two Child Policy in 2015, and the Three Child Policy in 2021.
\end{itemize}

\subsection*{Overall Commentary}
\begin{itemize}
    \item China has experienced massive population growth over the last few centuries, becoming the most populous country in the world.
    \item This growth has been shaped by a range of factors, including agricultural development, political change, wars, and population policies.
    \item Managing such a large population presents unique challenges, including ensuring sustainable development, providing adequate healthcare, and addressing the needs of an aging population.
\end{itemize}