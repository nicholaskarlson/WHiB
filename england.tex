\section{Population Growth in England: 1700 - 2020}

\subsection*{1700 - 1950 (Every 50 years)}
\begin{itemize}
    \item \textbf{1700}: Approximately 5.1 million.
    \item \textbf{1750}: Around 5.7 million.
    \item \textbf{1800}: Close to 8.3 million.
    \item \textbf{1850}: Approximately 16.8 million.
    \item \textbf{1900}: About 30.5 million.
    \item \textbf{1950}: Roughly 41.2 million.
\end{itemize}

\subsection*{Commentary 1700 - 1950}
\begin{itemize}
    \item \textbf{1700-1800}: Modest growth, influenced by agricultural developments and the early stages of industrialization.
    \item \textbf{1800-1900}: Significant growth due to the Industrial Revolution, urbanization, and improvements in medicine and sanitation.
    \item \textbf{1900-1950}: Continued growth, though impacted by two World Wars and the 1918 influenza pandemic.
\end{itemize}

\subsection*{1950 - 2020 (Every 10 years)}
\begin{itemize}
    \item \textbf{1950}: Approximately 41.2 million.
    \item \textbf{1960}: Around 43.4 million.
    \item \textbf{1970}: Approximately 46.0 million.
    \item \textbf{1980}: Close to 46.8 million.
    \item \textbf{1990}: Approximately 49.1 million.
    \item \textbf{2000}: About 49.8 million.
    \item \textbf{2010}: Roughly 52.2 million.
    \item \textbf{2020}: Estimated to be around 56.3 million.
\end{itemize}

\subsection*{Commentary 1950 - 2020}
\begin{itemize}
    \item \textbf{1950-1980}: Steady growth, with post-war recovery and economic expansion.
    \item \textbf{1980-2000}: Slower growth, reflecting broader demographic trends in developed countries.
    \item \textbf{2000-2020}: Continued, steady growth, with increases in life expectancy and immigration.
\end{itemize}

\subsection*{Overall Commentary}
\begin{itemize}
    \item England has experienced significant population growth over the last few centuries, particularly during the periods of industrialization and urbanization.
    \item The challenges associated with this growth have evolved over time, from managing urban crowding and public health in the 19th century, to addressing issues of sustainability, housing, and social services in the 21st century.
    \item The population growth rate has stabilized in recent decades, but the impacts of this growth continue to influence policy and planning in England.
\end{itemize}
