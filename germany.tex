\section*{Population Growth in Germany: 1700 - 2020}

\subsection*{1700 - 1950 (Every 50 years)}
\begin{itemize}
    \item \textbf{1700}: Estimated to be around 10 million. 
    \item \textbf{1750}: Approximately 16 million.
    \item \textbf{1800}: Around 24 million.
    \item \textbf{1850}: Close to 35 million.
    \item \textbf{1900}: Approximately 56 million.
    \item \textbf{1950}: Roughly 68 million (including East and West Germany).
\end{itemize}

\subsection*{Commentary 1700 - 1950}
\begin{itemize}
    \item \textbf{1700-1800}: The population of the regions constituting modern-day Germany grew significantly, although there were regional variations.
    \item \textbf{1800-1850}: Continued growth, influenced by agricultural improvements and the early stages of industrialization.
    \item \textbf{1850-1900}: Rapid population growth due to industrialization, urbanization, and improved living conditions.
    \item \textbf{1900-1950}: Population growth continued, despite the severe impacts of two World Wars. The aftermath of World War II left Germany divided into East and West.
\end{itemize}

\subsection*{1950 - 2020 (Every 10 years)}
\begin{itemize}
    \item \textbf{1950}: Roughly 68 million (including East and West Germany).
    \item \textbf{1960}: Around 72.5 million (including East and West Germany).
    \item \textbf{1970}: Approximately 78 million (including East and West Germany).
    \item \textbf{1980}: Close to 78.2 million (including East and West Germany).
    \item \textbf{1990}: Approximately 79.8 million (just after reunification).
    \item \textbf{2000}: About 82.2 million.
    \item \textbf{2010}: Roughly 81.8 million.
    \item \textbf{2020}: Estimated to be around 83 million.
\end{itemize}

\subsection*{Commentary 1950 - 2020}
\begin{itemize}
    \item \textbf{1950-1990}: Population growth in both East and West Germany, with rapid economic recovery and development after World War II. The country was reunified in 1990.
    \item \textbf{1990-2020}: The population showed signs of aging and a slight decline in the 2010s, but it has been somewhat offset by immigration, leading to a more diverse population.
\end{itemize}

\subsection*{Overall Commentary}
\begin{itemize}
    \item From 1700 to 2020, the population of Germany experienced significant growth, increasing from around 10 million to about 83 million.
    \item Industrialization in the 19th and early 20th centuries played a crucial role in population growth.
    \item The 20th century saw challenges including two World Wars and division during the Cold War, but also rapid economic development and population growth.
    \item The 21st century is characterized by an aging population and the impact of immigration on demographic trends.
\end{itemize}
