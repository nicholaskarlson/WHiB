\section*{Population Growth in Indonesia: 1700 - 2020}

\subsection*{1700 - 1950 (Every 50 years)}
\begin{itemize}
    \item \textbf{1700}: Estimated to be around 4.7 million.
    \item \textbf{1750}: Approximately 5.5 million.
    \item \textbf{1800}: Around 7 million.
    \item \textbf{1850}: Close to 10 million.
    \item \textbf{1900}: Approximately 37 million.
    \item \textbf{1950}: Roughly 70 million.
\end{itemize}

\subsection*{Commentary 1700 - 1950}
\begin{itemize}
    \item \textbf{1700-1800}: Slow population growth during this period, reflecting agricultural patterns and the impact of colonial rule.
    \item \textbf{1800-1900}: Significant population growth, influenced by increased agricultural production and the introduction of new crops.
    \item \textbf{1900-1950}: A period of rapid population growth, influenced by better health care, and stability, despite the impact of World War II and the struggle for independence.
\end{itemize}

\subsection*{1950 - 2020 (Every 10 years)}
\begin{itemize}
    \item \textbf{1950}: Approximately 70 million.
    \item \textbf{1960}: Around 88 million.
    \item \textbf{1970}: Approximately 120 million.
    \item \textbf{1980}: Close to 150 million.
    \item \textbf{1990}: Approximately 182 million.
    \item \textbf{2000}: About 206 million.
    \item \textbf{2010}: Roughly 238 million.
    \item \textbf{2020}: Estimated to be around 273 million.
\end{itemize}

\subsection*{Commentary 1950 - 2020}
\begin{itemize}
    \item \textbf{1950-1980}: Rapid population growth during this period, influenced by continued improvements in health care and living standards.
    \item \textbf{1980-2000}: Continued population growth, though at a slightly reduced rate compared to previous decades.
    \item \textbf{2000-2020}: Further population growth, with increased urbanization and improvements in education and health care, though the rate of population growth has started to slow down.
\end{itemize}

\subsection*{Overall Commentary}
\begin{itemize}
    \item From 1700 to 2020, Indonesia experienced substantial population growth, from 4.7 million to around 273 million.
    \item The country transitioned from a collection of sultanates and colonies to an independent nation.
    \item Indonesia faces challenges related to managing its large and diverse population, including issues related to urbanization, health care, education, and sustainable development.
\end{itemize}