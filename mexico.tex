\section{Population Growth in Mexico: 1700 - 2020}

\subsection*{1700 - 1950 (Every 50 years)}
\begin{itemize}
    \item \textbf{1700}: Estimated to be around 6 million.
    \item \textbf{1750}: Approximately 3.5 million (decline due to diseases and socio-economic conditions).
    \item \textbf{1800}: Around 5.5 million.
    \item \textbf{1850}: Close to 7.5 million.
    \item \textbf{1900}: Approximately 13.6 million.
    \item \textbf{1950}: Roughly 28.3 million.
\end{itemize}

\subsection*{Commentary 1700 - 1950}
\begin{itemize}
    \item \textbf{1700-1800}: Population decline due to epidemics, diseases, and harsh living conditions.
    \item \textbf{1800-1900}: Slow recovery and growth, despite political turmoil and the Mexican-American War.
    \item \textbf{1900-1950}: Significant growth, influenced by relative stability and economic development.
\end{itemize}

\subsection*{1950 - 2020 (Every 10 years)}
\begin{itemize}
    \item \textbf{1950}: Approximately 28.3 million.
    \item \textbf{1960}: Around 38.7 million.
    \item \textbf{1970}: Approximately 48.2 million.
    \item \textbf{1980}: Close to 68.3 million.
    \item \textbf{1990}: Approximately 81.2 million.
    \item \textbf{2000}: About 97.4 million.
    \item \textbf{2010}: Roughly 113.4 million.
    \item \textbf{2020}: Estimated to be around 126 million.
\end{itemize}

\subsection*{Commentary 1950 - 2020}
\begin{itemize}
    \item \textbf{1950-1980}: Rapid population growth, influenced by high birth rates and improved healthcare.
    \item \textbf{1980-2000}: Continued growth but at a slower rate, with efforts to promote family planning.
    \item \textbf{2000-2020}: Further slowing of growth rate, with increasing focus on addressing demographic challenges and sustainable development.
\end{itemize}

\subsection*{Overall Commentary}
\begin{itemize}
    \item Mexico’s population has experienced significant growth and change over the past few centuries.
    \item Challenges associated with this growth include urbanization, providing adequate services, and ensuring sustainable development.
    \item The country’s demographic profile has also evolved, with a trend towards an aging population and smaller household sizes.
\end{itemize}
