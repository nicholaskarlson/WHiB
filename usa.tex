\section{Population Growth in the USA: 1700 - 2020}

\subsection*{1700 - 1950 (Every 50 years)}
\begin{itemize}
    \item \textbf{1700}: Estimated to be around 250,000 (mostly Indigenous peoples and European settlers).
    \item \textbf{1750}: Approximately 1.2 million (significant growth due to immigration and natural increase).
    \item \textbf{1800}: Around 5.3 million.
    \item \textbf{1850}: Close to 23.2 million (boosted by westward expansion and immigration).
    \item \textbf{1900}: Approximately 76 million.
    \item \textbf{1950}: Roughly 151 million.
\end{itemize}

\subsection*{Commentary 1700 - 1950}
\begin{itemize}
    \item \textbf{1700-1800}: Rapid growth with immigration and expanding settlements.
    \item \textbf{1800-1900}: Explosive growth due to westward expansion, industrialization, and large waves of immigration.
    \item \textbf{1900-1950}: Continued growth, though impacted by the Great Depression and two World Wars.
\end{itemize}

\subsection*{1950 - 2020 (Every 10 years)}
\begin{itemize}
    \item \textbf{1950}: Approximately 151 million.
    \item \textbf{1960}: Around 179 million.
    \item \textbf{1970}: Approximately 203 million.
    \item \textbf{1980}: Close to 226 million.
    \item \textbf{1990}: Approximately 249 million.
    \item \textbf{2000}: About 281 million.
    \item \textbf{2010}: Roughly 309 million.
    \item \textbf{2020}: Estimated to be around 331 million.
\end{itemize}

\subsection*{Commentary 1950 - 2020}
\begin{itemize}
    \item \textbf{1950-1980}: Continued growth, marked by the baby boom and increasing diversity.
    \item \textbf{1980-2000}: Growth continued but at a slightly slower rate; the economy and technological advances were significant factors.
    \item \textbf{2000-2020}: Further growth, with increased focus on sustainable development, healthcare, and addressing social issues.
\end{itemize}

\subsection*{Overall Commentary}
\begin{itemize}
    \item The USA’s population has seen substantial growth and change over the centuries, becoming one of the most populous and diverse nations in the world.
    \item The country has faced and continues to face challenges related to its large and diverse population, including issues of inequality, healthcare, and sustainable development.
    \item The population is also aging, with implications for healthcare, social security, and the workforce.
\end{itemize}
